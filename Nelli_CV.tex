%------------------------
% Resume Template
% Author : Anubhav Singh
% Github : https://github.com/xprilion
% License : MIT
%------------------------

\documentclass[a4paper,20pt]{article}

\usepackage{latexsym}
\usepackage[empty]{fullpage}
\usepackage{titlesec}
\usepackage{marvosym}
\usepackage[usenames,dvipsnames]{color}
\usepackage{verbatim}
\usepackage{enumitem}
\usepackage{etoolbox}
\usepackage[pdftex]{hyperref}
\usepackage{fancyhdr}

\pagestyle{fancy}
\fancyhf{} % clear all header and footer fields
\fancyfoot{}
\renewcommand{\headrulewidth}{0pt}
\renewcommand{\footrulewidth}{0pt}

% \usepackage[maxbibnames=99,sorting=ydnt]{biblatex}
\usepackage[maxbibnames=99,sorting=ndymdt,doi=false,isbn=false,url=false,eprint=false]{biblatex}
\renewbibmacro{in:}{}

\DeclareSortingTemplate{ndymdt}{
  \sort{
    \field{presort}
  }
  \sort[final]{
    \field{sortkey}
  }
  % \sort{
  %   \field{sortname}
  %   \field{author}
  %   \field{editor}
  %   \field{translator}
  %   \field{sorttitle}
  %   \field{title}
  % }
  \sort[direction=descending]{
    \field{sortyear}
    \field{year}
    \literal{9999}
  }
  \sort[direction=descending]{
    \field[padside=left,padwidth=2,padchar=0]{month}
    \literal{99}
  }
  \sort[direction=descending]{
    \field[padside=left,padwidth=2,padchar=0]{day}
    \literal{99}
  }
  \sort{
    \field{sorttitle}
  }
  \sort[direction=descending]{
    \field[padside=left,padwidth=4,padchar=0]{volume}
    \literal{9999}
  }
}

% \bibliography{bibliography.bib}
\addbibresource{bibliography.bib}
\nocite{*}

% Adjust margins
\addtolength{\oddsidemargin}{-0.530in}
\addtolength{\evensidemargin}{-0.375in}
\addtolength{\textwidth}{1in}
\addtolength{\topmargin}{-.45in}
\addtolength{\textheight}{1in}

\urlstyle{rm}

\raggedbottom
\raggedright
\setlength{\tabcolsep}{0in}

% Sections formatting
\titleformat{\section}{
  \vspace{-10pt}\scshape\raggedright\Large
}{}{0em}{}[\color{black}\titlerule \vspace{-6pt}]

%-------------------------
% Custom commands
\newcommand{\resumeItem}[1]{
  \item\small{
    #1 \vspace{-2pt}
  }
}
\newcommand{\resumeSubItem}[2]{
  \item{
    \textbf{#1}{: #2 \vspace{-2pt}}
  }
  \vspace{-3pt}
}

\newcommand{\resumeItemWithoutTitle}[1]{
  \item\small{
    {\vspace{-2pt}}
  }
}


\newcommand{\resumeSubheading}[4]{
  \vspace{-1pt}\item
    \begin{tabular*}{0.97\textwidth}{l@{\extracolsep{\fill}}r}
      \textbf{#1} & #2 \\
    \end{tabular*}
    \begin{tabular*}{0.97\textwidth}{l@{\extracolsep{\fill}}r}
    #3 & \textit{#4} \\
    \end{tabular*}
    \vspace{-5pt}
}

\newcommand{\resumeSubsubheading}[2]{
  \vspace{-1pt}
    \begin{tabular*}{0.97\textwidth}{l@{\extracolsep{\fill}}r}
    #1 & \textit{#2}
    \end{tabular*}
    \vspace{-5pt}
}

\newcommand{\resumeEducation}[5]{
  \vspace{-1pt}\item
    \begin{tabular*}{0.97\textwidth}{l@{\extracolsep{\fill}}r}
      \textbf{#1} & #2 \\
      \end{tabular*}
      #3: \textit{#4}\\
      #3: \textit{#5}
    \vspace{-5pt}
}

\newcommand{\resumeAward}[3]{
  \vspace{-1pt}\item
    \begin{tabular*}{0.97\textwidth}{l@{\extracolsep{\fill}}r}
      \textbf{#1}{\ifstrequal{#3}{}{}{, #3}} & #2
    \end{tabular*}\vspace{-15pt}
}


\renewcommand{\labelitemii}{$\circ$}

\newcommand{\resumeSubHeadingListStart}{\begin{itemize}[leftmargin=*]}
\newcommand{\resumeSubHeadingListEnd}{\end{itemize}}
\newcommand{\resumeItemListStart}{\begin{itemize}}
\newcommand{\resumeItemListEnd}{\end{itemize}\vspace{-5pt}}

%-----------------------------
%%%%%%  CV STARTS HERE  %%%%%%

\begin{document}

%----------HEADING-----------------
\begin{tabular*}{\textwidth}{l@{\extracolsep{\fill}}r}
  \textbf{{\LARGE Kyle C. Nelli}}\\
  \href{https://github.com/knelli2}{Github: ~github.com/knelli2} &
   Email: \href{mailto:}{knelli@caltech.edu} \\
  \href{https://www.linkedin.com/in/kyle-nelli}{LinkedIn: linkedin.com/in/kyle-nelli} &
   Mobile:~~847-494-5028 \\
\end{tabular*}

%----------OBJECTIVE-----------------

\section{Objective}
Extremely dedicated software engineer and physics PhD student seeking a
future career opportunity or Summer 2025 internship at SideFX. I currently have
extensive experience in physical-based simulations, CPU and GPU programming in
C++, algorithm and program optimization, and an uncanny ability to uncover and
remove bugs in a large code base. I am very excited by the possibility of being
able to develop and improve Houdini at SideFX to support artists in their new
creations.

%-------------SKILLS-----------------
\vspace{5pt}
\section{Skills Summary}
	\resumeSubHeadingListStart
	\resumeSubItem{Languages}{~~~C/C++ 20, Python, CUDA, Bash, Perl, Mathematica}
	\resumeSubItem{Software}{~~~~~~VisIt, Paraview, Blender, SpECTRE, VSCode,
	\LaTeX, GNUPlot, Kokkos}
	\resumeSubItem{Tools}{~~~~~~~~~~~Git/GitHub, Docker/DockerHub, Make/Ninja, CMake, LLVM, GCC, GDB, HPCToolkit, SLURM}
  \resumeSubItem{Parallelism}{~~MPI, OpenMP, Charm++}
	\resumeSubItem{Platforms}{~~~~Linux (Ubuntu, Mint, CentOS, RedHat, Rocky), Windows, MacOS}
	\resumeSubItem{SuperComputers}{\\~~~~~~~~~~~~~~~~~~~~ Wheeler (CPU), Caltech HPC (CPU/GPU),
	Perlmutter (GPU) Frontera (CPU), Anvil (CPU), \\~~~~~~~~~~~~~~~~~~~~ Expanse (CPU),
  Bridges2 (CPU), NASA Pleiades (CPU/GPU), Blue Waters (CPU)} 
\resumeSubHeadingListEnd

\vspace{5pt}
\section{Experience}
  \resumeSubHeadingListStart
  \resumeSubheading{Lead Software Developer/Engineer}{May 2023 -
  Present}{\href{https://github.com/sxs-collaboration/spectre}{SpECTRE} (700k+
  lines of C++ 20)}{}
    \resumeItemListStart
  \resumeItem{Open-source software designed to run highly accurate simulations
  of black hole collisions and magnetized fluids on HPC, exascale, and GPU machines.}
  \resumeItem{Utilized task-based (asynchonous) parallelism to
  achieve \textbf{3x speedup} when solving partial differential equations on exascale
  computing resources.}
  \resumeItem{\textbf{Visualized $\sim$1TB} of output from simulations using Paraview and its
  Python scripting framework.}
  \resumeItem{Implemented detailed memory diagnostics which resulted in
  \textbf{reduction of memory usage by 5x}.}
  \resumeItem{Engineered and \textbf{streamlined automated simulation pipeline
  infrastructure} to greatly enhance user experience.}
  \resumeItem{Instructed undergraduate students on ``best coding practives''.}
  \resumeItem{\textbf{Organized week-long software tutorial} at
  collaboration-wide conference.}
  \resumeItem{Designed and \textbf{oversaw 10 student projects}. Mentored
  undergraduate, masters, and other doctoral students.}
    \resumeItemListEnd

  \resumeSubheading{Graduate Research Assistant}{November 2020 -
  Present}{Caltech}{}
    \resumeItemListStart
  \resumeItem{Updated and significantly reduced code complexity of control loops in
  MPI-based Spectral Einstein Code (SpEC).} 
  \resumeItem{Present and clearly communicate novel research at 7 different
  conferences.}
  \resumeItem{Teach and lecture for graduate-level physics course}
      \resumeItemListEnd

    \resumeSubheading{Undergraduate Researcher}{May 2018 - July 2020}{University of Illinois Department of Physics, REU}{}
    \resumeItemListStart
  \resumeItem{Wrote novel code in Python and C++ (27k+ lines) to automate visualization using
  VisIt software and Blue Waters supercomputer.}
  \resumeItem{Created visualizations of numerical simulations of black hole and
  compact star mergers that appear in \textbf{5 publications}.}
      \resumeItemListEnd

    \resumeSubheading
		{Undergraduate Research Internship}{May 2017–July 2017}
		{Argonne National Laboratory}{}
		\resumeItemListStart
        \resumeItem{Utilized Advanced Photon Source (X-rays) to record fuel injector spray patterns.}
        \resumeItem{Generated novel Python scripts to analyze experimental data for start of injection time; implemented visualizations with Blender software.}
		\resumeItemListEnd
\resumeSubHeadingListEnd

%-----------EDUCATION-----------------
\vspace{5pt}
\section{Education}
  \resumeSubHeadingListStart
    \resumeSubheading
      {California Institute of Technology (Caltech), CA}{August 2020 - Present}
      {Doctorate of Philosophy: \textit{Physics}}{}
    \resumeEducation
      {University of Illinois Urbana-Champaign, IL}{August 2016 - May 2020}
{Bachelor of Science}{Engineering Physics, Highest Honors}{Astronomy, Summa Cum
Laude and with High Distinction}
    \resumeSubHeadingListEnd

% %-----------AWARDS-----------------
% \section{Honors and Awards}
% \resumeSubHeadingListStart
%     \resumeAward{APS DGRAV Travel Grant}{April 2023,24}{\$300}
%     \resumeAward{ICERM Travel Grant}{August 2022}{\$840}
%     \resumeAward{David and Barbara Groce travel fund}{2022-2024}{\$500 per year}
%     \resumeAward{Rochus E. Vogt Graduate Fellowship}{Fall 2020 - Fall 2021}{\$36,500}
%     \resumeAward{Excellence in Physics Scholarship}{Spring 2020}{\$3,000}
%     \resumeAward{Anthony Research Scholarship}{Spring 2020}{\$1,000}
%     \resumeAward{Wyatt, Stanley Memorial Award}{Spring 2020}{\$700}
%     \resumeAward{University of Illinois Dean’s List}{August 2016 - May 2020}{Top 20\% in College of Engineering}
%     \resumeAward{Illinois Tool Works Scholarship}{August 2016 - May 2020}{\$1,500 per academic year}
%     \resumeAward{Phi Beta Kappa Honor Society}{2019}{Member}
%     \resumeAward{A.C. Anderson Undergraduate Research Award}{Summer 2018}{}

% \resumeSubHeadingListEnd

%---------PRESENTATIONS------------
\vspace{10pt}
\section{Presentations}
\resumeSubHeadingListStart
\resumeItem{``How to write code that somebody other than you can read and
understand'', LIGO SURF Software Seminar, June 27 2024, Pasadena, CA}
\resumeItem{``The SpECTRE CCE Module'', North American Einstein Toolkit
Workshop, June 3 2024, Baton Rouge, LA}
\resumeItem{``Horizon Tracking in SpECTRE with Task-Based Parallelism'', April
APS Meeting, April 3 2024, Sacramento, CA}
\resumeItem{``Horizon Tracking in SpECTRE with Task-Based Parallelism'', Pacific
Coast Gravity Meeting, March 1 2024, Santa Barbara, CA}
\resumeItem{``Cauchy-Characteristic Matching in SpECTRE'', April APS Meeting,
April 16 2023, Minneapolis, MN}
\resumeItem{``Cauchy-Characteristic Matching in SpECTRE'', Pacific Coast Gravity
Meeting, April 1 2023, Caltech, CA}
\resumeItem{``SpECTRE, Numerical Relativity Community Summer School 2022'',
Numerical Relativity Community Summer School, Aug. 11 2022, ICERM at Brown
University, MA}
\resumeSubHeadingListEnd

%---------PUBLICATIONS------------
\vspace{5pt}
\printbibliography[title=Publications]


\end{document}